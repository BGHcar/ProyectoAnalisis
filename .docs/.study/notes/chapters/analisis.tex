\chapterimage{./Pictures/algorithmic_analysis.png}
\chapter{Análisis algorítmico}

Estamos acostumbrados a manejar estructuras:
\begin{itemize}
	\item De control: if, else if, else, anidaciones.
	\item De flujo: for, while, repeat.
	\item De secuencia: Donde las instrucciones (accion $i$) se ejecutan sucesivamente sin omisiones (accion $i$ requiere accion $i-1$).
\end{itemize}

\section{Introducción}
Un algoritmo es el conjunto de procedimientos o funciones o bloques con estructuras de control, deben ser precisos y dar término.
Una función $func$ realizan una tarea y devuelven un valor como resultado o salida.
Un procedimiento $proc$ es un fragmento de código que no retorna un dato sino que trabaja mutando los datos del encabezado.
Los parámetros de un algoritmo son 03:
\begin{enumerate}
	\item Forma: Entrada, Salida o Mixta.
	\item Tipo: La estructura o tipo de dato.
	\item Nombre: Clave cual referencia la variable.
\end{enumerate}
% \newpage
\begin{example}[Costeo algorítmico]~
	\\

	Dado el procedimiento $P0$ en cada línea debemos costear: Número de operaciones elementales, Número de ejecuciones.
	\begin{multicols}{2}
		\begin{lstlisting}
1| proc P0(E int n):
2|     x = 100
3|     z = x + 5
4|     for (i = 1 to n) do
5|         for (j = 1 to n) do
6|             x = x + 1
        \end{lstlisting}
		\columnbreak
		\begin{enumerate}
			\item Llamado sin coste asociado.
			\item $1\times1$
			\item $2\times1$
			\item $3\times\sum_{i=1}^{n+1}1$
			\item $3\times\sum_{i=1}^{n}\sum_{j=1}^{n+1}1$
			\item $2\times\sum_{i=1}^{n}\sum_{j=1}^{n}1$
		\end{enumerate}
	\end{multicols}
	En este escenario al usarse un for se iterará siempre la misma cantidad, con ello estamos en un caso invariante o promedio.
	\\Nótese como siempre el número de ejecuciones en un ciclo se sigue por: $cabeza=cuerpo+1$.
	Eventualmente resolviendo las sumatorias obtenemos la función de eficiencia para el Coste en Complejidad Computacional Temporal \textit{(CC.T(n))}:
	$$
		T(n)=1\times1+2\times1+3\times(n+1)+3\times n\times(n+1)+2\times n\times n
	$$ $$
		T(n)=an^2+bn+c\in \Theta(n^2)
	$$
	Si realizamos el costeo en términos de la Complejidad Computacional Espacial \textit{(CC.S(n))} notamos como sólo se realizan operaciones elementales, todas con un coste constante $(c)$.
	$$
		S(n)=c\in\Theta(1)
	$$
\end{example}

La jerarquía de operaciones para cálculos computacionales es:
\begin{enumerate}
	\item Brackets.
	\item Powers.
	\item Products, Ratios.
	\item Relational operations.
\end{enumerate}

Realizado sobre una función:
\subsection{Sequential Search}
\begin{example}[Análisis búsqueda secuencial]~
	\\Deben definirse:
	\\\textbf{Precondiciones:} Lo que entra; Un arreglo de números enteros y un número entero.
	\\\textbf{Poscondiciones:} Lo obtenido y cómo se obtiene, además de excepciones; Índice del elemento (número entero) ingresado en el arreglo, si no existe retorna $-1$.

	\begin{multicols}{2}
		\begin{lstlisting}
1| func seq_search(
 |   E list[int] A[n], int x, S int p
 | ):
2|     int i = 1
3|     repeat:
4|         if A(i) == x:
5|             return i
6|         i++
7|     until i > n
8|     return -1
        \end{lstlisting}

		\columnbreak
		El mejor escenario o $T_{best}$ es que exista el elemento y sea el primero en la lista.
		\begin{enumerate}
			\item Sin coste.
			\item $1\times1$
			\item Sin coste.
			\item $1\times1$
			\item $1\times1$
			\item $0\times1$
			\item $0\times1$
			\item $0\times1$
		\end{enumerate}
	\end{multicols}

	$$
		T_{best}(n)=1\times1+1\times1+1\times1
	$$ $$
		T_{best}(n)=c\in O(1)
	$$
	No obstante aún faltan escenarios.
	\begin{multicols}{2}
		\begin{lstlisting}
1| func seq_search(
 |   E list[int] A[n], int x, S int p
 | ):
2|     int i = 1
3|     repeat:
4|         if A(i) == x:
5|             return i
6|         i++
7|     until i > n
8|     return -1
        \end{lstlisting}

		\columnbreak
		El peor escenario o $T_{worst}$ es que no exista el elemento.
		\begin{enumerate}
			\item Sin coste.
			\item $1\times1$
			\item $0\times n$
			\item $1\times n$
			\item $0\times n$
			\item $1\times n$
			\item $1\times n$
			\item $1\times1$
		\end{enumerate}
	\end{multicols}
	$$
		T_{worst}(n)= 1\times1+1\times n+1\times 1\times n+1\times1
	$$ $$
		T_{worst}(n)= an+b\in O(n)
	$$
\end{example}


\section{Probabilidad}
\subsection{Eventos deterministas}
\begin{definition}[Suceso elemental]
	Recoge información de todo el experimento
	\begin{example}[Dados]~
		\\Lanzar un dado genera 1 resultado o suceso elemental; Tras 10 resultados generamos un prorrateo, una razón.
	\end{example}
\end{definition}

\begin{definition}[Sucesos]
	\textbf{Seguro}: Donde la probabilidad del evento es 1, $P(S)=1$.
	\textbf{Imposible}: Donde la probabilidad del evento es 0, $P(S)=0$.
\end{definition}

\subsubsection{Probabilidad condicional}
Definida como $P(S|T)=\frac{P(S\cap T)}{P(T)}$

\begin{example}[Dados]~
	\\Tras lanzar 02 dados se tiene que $(S_1:D_1~\text{cae}~1)$, para $(S_2:D_2~\text{cae}~6)$ y el $(S_3:D_1+D_2\le4$. Se busca hallar la $P(S_1|S_3)$.
	\\Encontramos que $P(S_1)=6/36=1/6$, la $P(S_3)=6/36=1/6$ y que la $P(S_1\cap S_3)=3/36=1/12$.
	$$
		P(S_1|S_3)=\frac{3/36}{6/36}=1/2
	$$
\end{example}

\section{Principio de correctitud}
Un algoritmo es correcto si para cada posible entrada se termina con la salida correcta \textit{(hace lo que afirma hacer)}, hay 02 pruebas.


\begin{itemize}
	\item{Correctitud parcial}:
	      Si el algoritmo no hace \textit{halting} entonces se produce un resultado correcto \textit{(PIM)}.

	\item{Terminación}:
	      Si da término en finitud de pasos, independientemente a la entrada.
\end{itemize}



\subsection{Insertion Sort}
Input: Arreglo  $A = [a_1, \cdots a_n]$.
\\
Process: Indexado lineal e iterandos contiguos, ante $A_j > A_i$ activa subiteración; sobre-escribe valores $A_{j+1} = A_{j}$ para ser decreciente izquierda. Finalmente escribe $x$ en penúltima j-iteración.

\begin{lstlisting}
def insertion_sort(A: list[int | float]) -> list[int | float]:
    ''' Numeric list sorted by insertion '''
    for i in range(1, len(A)):
        x: float = A[i]
        j: int = i-1
        while A[j] > x and j >= 0:
            A[j + 1] = A[j]
            j -= 1
        A[j + 1] = x
    return A
\end{lstlisting}

\subsection{Correctitud}
\begin{definition}[Lazo invariante]
	Sentencia cual prueba la correctitud algorítmica, aplicada sobre variables.
	A demostrar:
	\begin{itemize}
		\item Inicialización: Cierto previo a iteración primera.
		\item Mantenimiento: Si es cierto previo a iteración, permanece cierto previo a próxima.
		\item Terminación: Retorna una propiedad cual muestra la correctitud algorítmica (elemento probatorio).
	\end{itemize}
\end{definition}

\begin{example}[En Insertion sort]
	Tomando $A$ en orden.
	\begin{itemize}
		\item Inicialización: El caso de $A$ tamaño 1 es trivial.
		\item Mantenimiento: Cada elemento es menor a su siguiente.
		\item Terminación: En $i=n+1$ permanecen los elementos originales de $A$ \textbf{ordenados}.
	\end{itemize}
\end{example}

\subsection{Función de eficiencia}

Contar la función más ejecutada \textit{(permite saber el funcionamiento algorítmico)}, es la posible ejecución más anidada.


\begin{example}[Iteraciones múltiples]
	Un ejemplo práctico.

	\begin{lstlisting}
    def triple_for(n: int) -> int:
        x: int = 0
        for i in range(n):
            for j in range(n):
                for k in range(n):
                    x += 1
        return x
    \end{lstlisting}

	Con ello se buscará el comportamiento asintótico más representativo para el código mencionado.
	Para trabajar ciclos es necesaria la representación mediante sumatorias, desde la más interna a la más externa, donde tras resolver el más interno se pasa al más
	externo hasta dejarse como una función de tendencia.

	Supóngase $n=3$ entonces lo primero es contarse cada ejecución en k:
	Haciendo las iteraciones:
	$$triple\_for\begin{cases}
			~i=1:\quad j=(1,2),\quad k=(1,2)          \\
			~i=2:\quad j=(1,2,3),\quad k=(1,2)(1,2,3) \\
			~i=3:\quad j=(1,2,3,4),\quad k=(1,2)(1,2,3)(1,2,3,4)
		\end{cases}$$

	\begin{remark}
		Tenemos en total contando únicamente el número de ejecuciones en $k$, representable por la fórmula:
		$$
			f(n)=\sum_{i=1}^n i^2+i-2 \\
		$$ $$
			=\sum_{i=1}^n i^2 + \sum_{i=1}^n i - \sum_{i=1}^n 2
		$$ $$
			=\frac{n(n+1)(2n+1)}6+\frac{n(n+1)}2-2n
		$$
	\end{remark}
	Es apreciable que tras realizar la distribución de términos $n$ se obtiene $\frac13n^3+\cdots-\frac43n$ como el de mayor grado con otros términos de menor grado. Es así que podemos decir eventualmente tenemos un comportamiento asintótico de $n^3$, expresado \textbf{en la mejor forma} como $\Theta(n^3)$.
\end{example}

\section{Limites}
Tenemos como unos casos especiales que aplican a los límites:
$$
	\lim_{n\to\infty}\br{\frac1n}=0
$$
$$
	\lim_{n\to\infty}\br{\frac n{n+1}}=1
$$
$$
	\lim_{n\to\infty}(\sqrt[n]{a})=0;\quad a>0
$$
$$
	\lim_{n\to\infty}(\sqrt[n]{n})=1
$$
$$
	\lim_{n\to\infty}\br{\frac{ln[n]}{x^n}}=0;\quad(a>0)\land (b>0)
$$
$$
	\lim_{n\to\infty}\br{1+\frac1n}^n=e
$$
$$
	\lim_{n\to\infty}\br{\frac{Sin(x)}x}=1
$$
$$
	\lim_{n\to\infty}ATan(x)=\frac{\pi}2
$$
$$
	\lim_{n\to\infty}ASec(x)=\frac{\pi}2
$$
$$
	\lim_{x\to\infty}f(x)=L
	\implies
	\lim_{n\to\infty}f(n)=L
$$

\section{Series}
Generalización sobre la noción de suma sobre una función $f(n)$ acotada superior e inferiormente.

\subsection{Propiedades}
\begin{theorem}[Adiciones y sustracciones]
	$$
		\sum_{i=0}^n[f(i)\pm g(i)\pm\cdots\pm z(i)]
		= \sum_{i=0}^nf(i)\pm \sum_{i=0}^ng(i)\pm\cdots\pm\sum_{i=0}^nz(i)
	$$
\end{theorem}

\begin{theorem}[Constantes]
	$$ \sum_{i=0}^nc\cdot f(i) = c\cdot\sum_{i=0}^nf(i) $$
\end{theorem}

\begin{theorem}[Cambio de índice]
	$$ \sum_{i=0}^nf(i) = \sum_{i=k}^{n+k}f(i-k) $$
\end{theorem}

\begin{theorem}[Índice a 0]
	$$ \sum_{i=a}^bf(i) = \sum_{i=0}^{b-a}f(i+a) $$
\end{theorem}

\begin{theorem}[Partición]
	$$ \sum_{i=0}^nf(i) = \sum_{i=0}^cf(i)+\sum_{i=c+1}^nf(i) $$
\end{theorem}

\begin{theorem}[Dobles]
	$$
		\sum_{i=0}^p\sum_{j=0}^qf(i,j)
		\equiv \sum_{j=0}^q\sum_{i=0}^pf(i,j)
	$$
\end{theorem}

\subsection{Series comúnes}

\begin{definition}[Aritmética]
	Forma general
	$$ \sum_{i=1}^ni = \frac{n(n+1)}2 $$
	\begin{fact}[Orden]
		$O(n^2)$
	\end{fact}
\end{definition}

\begin{definition}[Geométrica]
	Forma general
	$$ \sum_{i=0}^nar^i = \frac{a(1-r^{n+1})}{1-r}\quad r\ne1 $$
	$a$: Primer término. $r$: Razón común.
	\begin{fact}[Orden]
		$|r|>1:O(r^n);\quad|r|<1:O(1)$
	\end{fact}
\end{definition}

\begin{definition}[Base 2]
	Forma general
	$$ \sum_{i=0}^nar^i = \frac{a(1-r^{n+1})}{1-r};\quad r\ne1 $$
	$a$: Primer término. $r$: Razón común.
	\begin{fact}[Orden]
		$|r|>1:O(r^n);\quad|r|<1:O(1)$
	\end{fact}
\end{definition}

\begin{definition}[Potencias]
	Forma general
	$$ \sum_{i=1}^ni^k$$
	Solución indefinida;
	\begin{remark}Generalización alterna $k=1$:
		$$\sum_{i=0}^nai+b=\frac{(an+2b)(n+1)}2$$
	\end{remark}
	$$
		k=2\implies\frac{2n^3+3n^2+n}6;\quad
		k=3\implies\frac{n^4+2n^3+n^2}4;\quad
	$$ $$
		k=4\implies\frac{6n^5+15n^4+10n^3-n}{30};\quad
		k=5\implies\frac{2n^6+6n^5+5n^4-n^2}{12};
	$$ $$
		k=6\implies\frac{6n^7+21n^6+21n^5-7n^3+n}{42};
	$$ $$
		k=7\implies\frac{3n^8+12n^7+14n^6-7n^4+2n^2}{24};
	$$ $$
		k=8\implies\frac{10n^9+45n^8+60n^7-42n^5+20n^3-3n}{90};
	$$ $$
		k=9\implies\frac{2n^{10}+10n^9+15n^8-14n^6+10n^4-3n^2}{20};
	$$ $$
		k=10\implies\frac{6n^{11}+33n^{10}+55n^9-66n^7+66n^5-33n^3+5n}{66}\dots
	$$

	\begin{fact}[Orden]
		$O(n^{k+1})$
	\end{fact}
\end{definition}

\begin{definition}[Armónica]
	Forma general
	$$ \sum_{i=1}^n\frac1i$$
	Solución indefinida;
	$$ \ln n+\gamma.\quad\gamma:\text{Euler-Mascheroni}=-\int_0^\infty e^{-x}\ln x~dx $$
	\begin{fact}[Orden]
		$O(\ln n)$
	\end{fact}
\end{definition}


\begin{definition}[Logarítmica]
	Forma general
	$$ \sum_{i=1}^n\log i \approx
		n\log n-n$$
	\begin{fact}[Orden]
		$O(n\log n)$
	\end{fact}
\end{definition}

\subsection{Productorias}

\begin{definition}[Factorial]
	$$
		\prod_{i=1}^ni
		=n!
	$$
\end{definition}

\begin{definition}[Constante]
	$$
		\prod_{i=1}^nk
		=k^n
	$$
\end{definition}

\begin{definition}[Generalizada]
	$$
		\prod_{i}^n a_{i+1}
		=k^n\prod_{i=1}^ni
	$$
\end{definition}

\begin{definition}[Escalar]
	$$
		\prod_{i=1}^nki
		=k^n\prod_{i=1}^ni
	$$
\end{definition}

\subsubsection{Propiedades telescópicas}

\begin{definition}[Generalizada]
	$$
		\prod_{i}^n a_{i+1}
		=k^n\prod_{i=1}^ni
	$$
\end{definition}

\subsubsection{Correlación}
\begin{definition}
	$$
		\lg \prod_{i=1}^n a_i
		= \sum_{i=1}^n\lg a_i
	$$
	$$
		\prod_{i=1}^n a_i
		= 2^{\sum_{i=1}^n\lg a_i}
	$$
\end{definition}

\subsection{Límites}
Es fundamental conocer los límites de cualquier función matemática para establecer relaciones próximamente en notaciones asintóticas. Recordemos sus propiedades:

$$\begin{array} {|r|r|}
		\hline  \text{Constantes}
		 & \quad\lim_{x\to c}k
		=k    \quad                                        \\
		\hline  \text{Identidad}
		 & \quad\lim_{x\to c}x
		=c    \quad                                        \\
		\hline  \text{Escalar}
		 & \quad\lim_{x\to c}kf(x)
		=k\lim_{x\to c}f(x)    \quad                       \\
		\hline  \text{Exponente}
		 & \quad\lim_{x\to c}x^p
		=c^p;    \quad r\ge0\quad                          \\
		\hline  \text{Adición}
		 & \quad\lim_{x\to c}[f(x)+g(x)]
		=\lim_{x\to c}f(x)+\lim_{x\to c}g(x)    \quad      \\
		\hline  \text{Substracción}
		 & \quad\lim_{x\to c}[f(x)-g(x)]
		=\lim_{x\to c}f(x)-\lim_{x\to c}g(x)    \quad      \\
		\hline  \text{Producto}
		 & \quad\lim_{x\to c}[f(x)\times g(x)]
		=\lim_{x\to c}f(x)\times\lim_{x\to c}g(x)    \quad \\
		\hline  \text{Razón}
		 & \quad\lim_{x\to c}[f(x)\div g(x)]
		=\lim_{x\to c}f(x)\div\lim_{x\to c}g(x);
		\quad \lim_{x\to c}g(x)\ne0                        \\
		\hline  \text{Potencia}
		 & \quad\lim_{x\to c}f(x)^{g(x)}
		=\lim_{x\to c}f(x)^{\lim_{x\to c}g(x)};
		\quad f(x)>0                                       \\
		\hline  \text{Logaritmo}
		 & \quad\lim_{x\to c}[\log f(x)]
		=\log[\lim_{x\to c}f(x)]    \quad                  \\
		\hline  \text{Radical}
		 & \quad\lim_{x\to c}[\sqrt{f(x)}]
		=\sqrt{\lim_{x\to c}f(x)}    \quad                 \\
		\hline
	\end{array}$$

Supóngase que se tiene una sucesión tal que tras hallar su \textit{n-ésimo} término $\lim_{n\to\infty}=L$

Los casos especiales son los siguientes:

$$
	\lim_{n\to\infty}\br{\frac1n}=0
$$ $$
	\lim_{n\to\infty}\br{\frac n{n+1}}=1
$$ $$
	\lim_{n\to\infty}(\sqrt[n]{a})=0;\quad a>0
$$ $$
	\lim_{n\to\infty}(\sqrt[n]n)=1;\quad n>0
$$ $$
	\lim_{n\to\infty}\br{\frac{ln[n]}{x^n}}=0;\quad a,b>0
$$ $$
	\lim_{n\to\infty}\br{1+\frac1n}^n=e;
$$ $$
	\lim_{n\to\infty}\br{\frac{Sin(n)}n}=1
$$ $$
	\lim_{n\to\infty}(ATan(x))=\lim_{n\to\infty}(ASec(x))=\frac{\pi}2
$$

\subsection{Criterios de convergencia}
Es fundamental conocer dada una serie su valor finito convergente.
\subsubsection{Criterio suficiente de divergencia}
Si $\lim_{n\to\infty}a_n$ no existe, o si $\lim_{n\to\infty}a_n\ne0$ entonces la Serie diverge.
~\\
Si el $\lim_{n\to\infty}a_n=0$ entonces \textbf{NO} se concluye nada.
~\\
Si $\sum a_n$ sabemos \textit{converge} entonces $\lim_{n\to\infty}(a_n=0)$.
